\chapter*{Abstract}
\addcontentsline{toc}{chapter}{Abstract}

    La comunicazione visiva è in molti casi più diretta ed immediata rispetto alla comunicazione verbale: disegni, foto e mappe sono esempi di frasi visive che necessitano di un contesto per essere descritte in modo naturale.
    \newline
    In questa tesi presento TiveJS, un'estensione della piattaforma \href{https://www.draw.io/}{draw.io}, che sfrutta  simboli e  definizioni sematiche per il riconoscimento dei linguaggi diagrammatici e la traduzione di questi in altri linguaggi.
    Il tool applica  delle definizioni semantiche  ad un diagramma e restituisce una traduzione di quest'ultimo. La traduzione avviene attraverso due fasi principali: il riconoscimento del grafo e l'applicazione delle definizioni.
    \newline
    Il mio lavoro di tesi si basa su strumenti precedentemente sviluppati: LoCoModeler e Tive. Precedentemente suddiviso in lato client e lato server, Tive è stato re-implementato completamente in JavaScript, prendendo il nome di TiveJS, eliminando così la necessità del server.
