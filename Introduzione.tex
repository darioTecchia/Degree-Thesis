\chapter{Introduzione}

    La comunicazione fra individui avviene in svariati modi, ad esempio attraverso il linguaggio verbale ed il linguaggio visivo (o linguaggio visuale).
    \newline
    Un linguaggio visuale non è altro che una forma di comunicazione, detta comunicazione visuale, che fa uso di simboli grafici o immagini. Simboli grafici, immagini e mappe sono esempi di elementi utilizzati all'interno della comunicazione visiva (o comunicazione visuale) che necessitano di un contesto per essere descritte in modo naturale. Spesso quest'ultima risulta essere molto più immediata e di facile comprensione rispetto alla tradizionale comunicazione verbale coposta di lettere e parole.
    \newline
    In questa tesi presento \textbf{TiveJS}, un'estensione della piattaforma \href{https://www.draw.io/}{draw.io}, che sfrutta  simboli e  definizioni sematiche per il riconoscimento dei linguaggi diagrammatici e la traduzione di questi in altri linguaggi.

    \section{Motivazioni}
        La piattaforma già esistente, LoCoMoTiVE, si basa su un meccanisco client-server.
        \newline
        Il client è formato dalla piattaforma draw.io, opportunamente modifica, per la creazione di sentenze visuali.
        Il server è stato implementato in Java utilizzando i servlet per il riconoscimento e la traduzione delle sentenze visuali.
        Il funzionamento è molto semplice, il client esegue una chiamata HTTP di tipo POST contenute al suo interno un grafo o un diagramma, creato attraverso l'utilizzo di simboli ad hoc, in formato XML.~Una volta ricevuta la sentenza visuale, il server la interpreta applicando le definizioni per poi restituire la traduzione semantica oppure dei messaggi di errore.
        \newline
        Le motivazioni che ci hannno portato alla creazione di un nuovo tool sono varie: rendere l'applicazione più scalabile e più veloce limitando l'interazione con il server a semplici accessi a pagine statiche; l'aggiornamento di TiVe all'ultima versione di draw.io.
        Essendo il core di TiveJS scritto completamente in JavaScript ora si integra perfettamente con la piattaforma estesa e con la manipolazione del grafo.
        Le definizioni dei linguaggi ora sono definite in formato JSON rendendo ancora più alta l'interoperabilità dei sistemi. 

    \section{Organizzazione della Tesi}
        Nel capitolo 2 illustrerò i lavori correlati al mio progetto di tesi.
        Nel capitolo 3 tratterò dei linguaggi visuali e dei loro componenti fondamentali. Nel capitolo 4 parlerò del Local Context e delle corrsipondenti specifiche sintattiche e semantiche. Nel capitolo 5 introdurrò il risultato del mio lavoro di tesi, TiveJS e le sue funzioni, e illustrerò  i dettagli dell' implementazione e le tecnologie usate. Nel capitolo 6 illustrerò vari casi d'utilizzo da me studiati. Nel capitolo 7, presenterò possibili sviluppi futuri dell'applicazione e le conclusioni.